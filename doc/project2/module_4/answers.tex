\section{Overall structure and key elements of our Expectimax system}
For this module we started writing in Python, as recommended by the course
staff. However, maybe due to the inherit nature of Python, or due to mistakes
in our first implementation we got pretty bad results using MiniMax with depth 3.
Increasing the depth was slow, and we thought that the heuristic was smart enough
that we should get tiles higher than the 512, and sometimes 1024 we were getting.

Therefore we decided to look for alternatives. We considered Java, but did not
want to rewrite our GUI. We could also have made a command line application in
Java, as we knew the interface would be fairly simple with just giving in a depth
and board, and getting back an integer representing a move. We feared that the
cost of shelling to CLI application, and starting the JVM for every move would
be a bottleneck.

So in the end we decided to use the built-in Python library ctypes, to interface
with a Expectimax implementation in C. We did not have a ton of experience using
C, so there are probably many more optimizations that could be done. Mostly we
ported our Python code line for line, and then a mixture of Valgrind to find
which functions took the most time, and just plain old compiler optimizations
to let GCC do our work for us. With this implementation we can go to depth 6
in Expectimax, and if we generate C instead of 2C chance successors the
movements are quite quick, in the order of 200ms. However, since we spent most
of our time optimizing the weight in the evaluation function when the c program
was somewhat slower, they are most suitable for 2C successors and only using depth 6.
This actually gives better results than searching deeper.

The code functions by having a Qt4 GUI written in Python 2.7, which
manages the application window, and a 2048 Widget that draws the board.
When the users presses play a QThread worker object is initialized which
manages the Expectimax search. It has a Play2048Player which keeps track of
Play2048States. Every move it sends the board to the widget which draws it.
The state contains the board, represented as a List(16). Originally this
state inherited a generic AdversialSearchState, but since the adversial
search in the final code is implemented entirely in C, this extra
genericity offered no benefit. The Play2048Player has a search field, which
contain an instance of the class the implements an adversial search. This
search only has to contain a decision(board) method which takes a board
as an argument and returns a move representing the best possible action
given the current board.

In our final implementation the search in Play2048Player is an instance of
ExpectimaxC. This is a simple proxy, which links to c .so-library and delegates
to the decision function implemented there. We had some interfacing issues between
C and Python, so the function signature in C is uglier than it should have to
be taking each tile in the board as a separate argument.

\section{Heuristics}

The decision function in the C implementation uses ExpectiMax to find the best move.
The evaluation function used has implemented ideas of smoothness, the highest tile
on the current board, the amount of free tiles, the placement of the highest tile,
be in a corner, on an edge or in the middle, and the monotonicity of the board.
The constants in this equation can be supplied from the Python side, so we
could create a test runner in Python which ran the search supplying different
values for these constants.

The two main ideas in this heuristic, aside from giving penalty to full board, and
misplaced high tiles are smoothness and monotonicity. Smoothness gives penalty
for the difference between neighbouring tiles. This means that neighbouring tiles
which are close in value are preferred to stay close together. The value of this is
that tiles can often be chained together in long tiles if we have for instance
1024, 512, 218 and 126 on the bottom row. Getting a new 126 above our existing 126
means we can get => 1024, 512, 218, 218 => 1024, 512, 512, x => 1024, 1024, x, x =>
2048, x, x, x. The smoother the board, the better the chance of this happening.

The other idea of monotonicity is checked on both the x-axis and the y-axis, and
the heuristic contributions of these are summed together. It checks that all the tiles
on both the x-axis and the y-axis are sorted. This also gives the advantage of
creating chains that can be merged together, and penalizes low tile values from
entering "behind" large tiles.

Removing the highest tile from a save corner, and having a low value get trapped
inside several large ones is the most common way of ending the game. Therefore
almost all our contributions to the heuristic focueses on this not happening.
