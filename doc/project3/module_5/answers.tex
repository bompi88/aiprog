%%
% **Deliverables**
%   1 - A 3-page report ( 5 points) that:
%     • Describes FIVE different ANNs that were constructed in Theano,
%       using different combinations of hidden layers, sizes of hidden
%       layers (in terms of the number of neurons that they contain),
%       and activation functions.
%     • Compares the training and testing performance of all FIVE ANNs,
%       with convincing statistics used to determine which network
%       performs best.

\section{Constructing Artifical Neural Networks for the MNIST data set}
This report describes five different artifical neural networks (ANNs)
that we have constructed using Theano during this project. They have
been trained and tested on the MNIST data set.
(\textcolor{red}{TODO}: Ref [http://yann.lecun.com/exdb/mnist/])
This is a set of
hand-written images of the digits 0-9.

\subsection{Theano construction}

Inspiration: (\textcolor{red}{TODO}: Ref [http://deeplearning.net/tutorial/logreg.html])

\subsection{The five ANNs}

%% 
% These 5 must then be rigorously tested by running them at least 20
% times each on the complete MNIST training and test sets. That is, for
% a given net, you must perform the following steps 20 times:
% backpropagation training followed by testing on both the original
% training data and the test data. All results must be recorded and
% summarized as part of a statistically legitimate comparison between
% the 5 designs. That comparison constitutes the core of the report
% for this module.

- Description
- Statistics

