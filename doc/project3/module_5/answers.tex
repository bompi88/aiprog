%%
% **Deliverables**
%   1 - A 3-page report ( 5 points) that:
%     • Describes FIVE different ANNs that were constructed in Theano,
%       using different combinations of hidden layers, sizes of hidden
%       layers (in terms of the number of neurons that they contain),
%       and activation functions.
%     • Compares the training and testing performance of all FIVE ANNs,
%       with convincing statistics used to determine which network
%       performs best.

\section{Constructing Artifical Neural Networks for the MNIST data set}
This report describes five different artifical neural networks (ANNs)
that we have constructed using Theano during this project. They have
been trained and tested on the MNIST data set.
\footnote{http://yann.lecun.com/exdb/mnist/} This is a set of
hand-written images of the digits 0--9.

\subsection{Theano construction}

Due to difficulties with module 6, we did not have much time left for report
writing, we have included what little we could.

\subsection{The five ANNs}

%% 
% These 5 must then be rigorously tested by running them at least 20
% times each on the complete MNIST training and test sets. That is, for
% a given net, you must perform the following steps 20 times:
% backpropagation training followed by testing on both the original
% training data and the test data. All results must be recorded and
% summarized as part of a statistically legitimate comparison between
% the 5 designs. That comparison constitutes the core of the report
% for this module.

Below is a table of our five ANNs. They all have an input layer of 784, and
an output layer of 10. We have used the theano Sigmoid function as the
activation function in all hidden layers. We have used the sum of squared
errors as the error function.

We have also included the testing data we used in determining that net 5
was the one we would use at the demo.

\begin{center}
  \begin{tabular}{ | c | c | c | c | c | c | }
    \hline
    \# & Hidden layer sizes & Learning rate & Training set & Testing set & Demo 100 \\ \hline
    1 & [150] & 0.1 & 99.75\% & 98.22\% & 91\% \\ \hline
    2 & [150] & 0.2 & 99.71\% & 97.99\% & 93\% \\ \hline
    3 & [200, 40] & 0.1 & 99.84\% & 98.24\% & 91\% \\ \hline
    4 & [40] & 0.1 & 98.93\% & 96.19\% & 85\% \\ \hline
    5 & [200, 40] & 0.2 & 99.83\% & 98.23\% & 93\% \\
    \hline
  \end{tabular}
\end{center}
