%% The report will be no longer than 3 pages and will include the following:
% 1. Verification of the generality of your A* algorithm. Any aspect of A* that 
% is specifically geared toward CSP and GAC should be in subclasses, not the 
% core A* classes.
% 2. Verification of the generality of your A*-GAC algorithm. Any aspect of 
% A*-GAC that is specifically geared toward VC should be in subclasses, not the 
% core code. You will be reusing A*-GAC on other assignments, so the core should
% be easily extendable via subclassing.
% 3. Verification of a clean separation between your Constraint Network (CNET) 
% and the VIs and CIs of the search states, as shown in Figure 4.
% 4. A clear explanation (with small code segments) of: a) the manner in which 
% your system creates code chunks on the fly, or b) the general workings of your 
% hand-built interpreter for executing constraints in a canonical form, or c) 
% your work-around for avoiding both code chunks and a hand-built interpreter.

\section{Central aspects of your A*-GAC program}
\subsection{Generality of the A*}
- Subclassing the BestFirstSearch class
- Subclassing the SearchState class
- 
% \lstinputlisting[emph={make_function}]{module_2/code_snippets/make_functions.py}

\subsection{Generality of the A*-GAC}
in the subclasses, overriding the create\_root\_node method to run gac.initialize() and gac.domain\_filtering(), as seen in code snippet \ref{code:create_root}.

\lstinputlisting[emph={create_root},label={code:create_root},caption={Generation of root node.}]{module_2/code_snippets/create_root.py}

To utilize the power of the GAC, the GAC has to be rerun on each generated successor in the overidden method `generate\_all\_successors', before it gets added to the open heap. Only successors which either got solved by the GAC, or did not get one or more of their domains reduced to an empty domain, is considered as a valid child and should be expanded. The `gac.rerun()' returns True if the successor is still a viable state after the GAC-reduction.

\lstinputlisting[label={code:successor_generation},caption={Generation of vertex successor states.}]{module_2/code_snippets/successor_generation.py}

\subsection{CNET, VIs and CIs}
- Constraint Network (CNET)\\
- constraint instances (VI)\\
Generates runnable expressions, as in this problem: \( v1 != v3 \), where \(v1\) and \(v2\) represent the vertices. In previous example, vertex \(v1\) and \(v2\) should not have the same color. The variable instances (CI) is being stored as list of strings in each constraint, as involved variables. These variables are used to retrieve the domains from a dictionary that is initialized at `gac.initialize()'.



\subsection*{Heuristics}
- selection of successors\\
- A*-heuristics

\subsection{Evaluation procedure of the GAC}
\lstinputlisting[label={code:make_function},caption={Generation of lambda expressions to evaluate the constraints.}]{module_2/code_snippets/make_functions.py}