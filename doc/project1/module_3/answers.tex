%% A 3-page report (5 points) that:
% 1. Clearly documents the representations (variables, domains and constraints)
% that you devised for solving nonograms.
% 2. Explains the heuristics used for this problem. Note that heuristics appear 
% in at least two places in A*-GSP: a) in A*’s traditional h function, and b) 
% in the choice of a variable on which to base the next assumption. Both (and 
% others, if relevant) should be mentioned in the report.
% 3. Briefly overviews the primary subclasses and methods needed to specialize 
% your general-purpose A*-GAC system to handle nonograms.
% 4. Mentions any other design decisions that are, in your mind, critical to 
% getting the system to perform well.

\section{Central aspects of the Nonogram implementation}
\subsection{Variables, domains, constraints}
For our representation of a Nonogram we have chosen to
use the rows and columns as variables. An idealized pseudocode
version can be seen below. The domain of each variable consists
of all the possible permutations of strings where 1 represents
marked and 0 represents unmarked in the final Nonogram.

If the amount of columns is 5, and a row has the set of segment
sizes \({1 2}\) our generated domains for this row rn will be
$ ['10110', '01011', '10011'] $.

As shown in the code below, we will do lookup in these strings
crosschecked with a column in cell c, to check whether the constraint
is satisfied. This is similiar to what the module text describes in
\emph{2.2 Using an Aggregate Representation} and shows in Figure 3.

As an example if rn above was r1 (zero-indexed), and we had a column
c3, we can clearly see from the example that c3[1] must be 1, as all the
valid domains for r1 contain 1 at the fourth position. So given domains
for domains[c3] = ['1010', '1100', '0011'], our implementation of revise
would limit the domain of c3 to ['1100'].

\lstinputlisting[emph={Nonogram}]{module_3/code_snippets/nonogram.py}

\subsection{Heuristics}
- H\-function\\
- Selection of next focus variable

\subsection{The implementation}

\subsection{Other aspects}
